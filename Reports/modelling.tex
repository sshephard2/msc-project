\section{Modelling}

In our presentation of technologies relevant to the ticketing system use case, we have discussed criteria to evaluate each of them individually.  In a distributed system we need to examine the end to end impacts.  What are the optimum relative numbers of Web servers and Worker Applications for a given demand? Does removing a bottleneck at the Web or Worker Application layer shift the stress to the middleware, or the database?  If we accept that some levels of demand cannot be met on a limited budget, and that some components will no longer meet the required throughput, how do we determine the impact on the remainder of the system?

\paragraph{Process Algebra.} Process Algebras (such as PEPA or TIPP \cite{gotz1993multiprocessor}) allow us to model throughput in interdependent processes, with a mixture of independent and shared actions operating at different rates.  Each of our components can be described in this way, and queues have already been extensively modelled in PEPA \cite{thomas1997using}.  The nature of process algebra as a mathematical language also means that it is possible to build a model of a whole system by composition of the component models.

\begin{figure}
\caption{PEPA queue model}
\centering
% Automatically generated by PEPA2Latex
% --begin
\begin{displaymath}
	\begin{array}{rcl}
%[0.0ex]		
\mathit{Web} & \rmdef & (\mathit{request},\mathit{r}).\mathit{Web}\\
		\mathit{Worker} & \rmdef & (\mathit{service},\mathit{s}).\mathit{Worker}\\
		\mathit{Queue_{0}} & \rmdef & (\mathit{request},\mathit{r}).\mathit{Queue_{1}}\\
		\mathit{Queue_{1}} & \rmdef & (\mathit{service},\mathit{s}).\mathit{Queue_{0}}\\
[0.0ex]		\multicolumn{3}{l}{\mathit{Web}\sync{request}\mathit{Queue_{0}}[N]\sync{service}\mathit{Worker}}\\
[0.0ex]	\end{array}
\end{displaymath}
% --end
\end{figure}

\paragraph{CloudSim.}  CloudSim \cite{calheiros2011cloudsim} is a Java framework for developing cloud datacentre simulations.  Much of it is concerned with modelling the efficient running of that infrastructure, for example the power usage, but it also includes utilisation models and may be useful for predicting the effect of elastic scaling.

CloudSim simulations require Java development for creation and modification, which is an overhead in building the models but offers more flexibility in applying them.  Process Algebra has closed-form solutions, though there is a PEPA Workbench tool \cite{gilmore1994pepa} that allows PEPA specifications to be parsed and run like programs, aiding experimentation on a range of action rates by automating repetitive calculations.  Both currently have their place as they predict different quantities of interest.
